\documentclass{article}

\usepackage{arxiv}

\usepackage[utf8]{inputenc} % allow utf-8 input
\usepackage[T1]{fontenc}    % use 8-bit T1 fonts
\usepackage{lmodern}        % https://github.com/rstudio/rticles/issues/343
\usepackage{hyperref}       % hyperlinks
\usepackage{url}            % simple URL typesetting
\usepackage{booktabs}       % professional-quality tables
\usepackage{amsfonts}       % blackboard math symbols
\usepackage{nicefrac}       % compact symbols for 1/2, etc.
\usepackage{microtype}      % microtypography
\usepackage{graphicx}

\title{Missing links and the topological robustness of food webs}

\author{
    Anubhav Gupta
    \thanks{Corresponding author}
   \\
    Department of Evolutionary Biology and Environmental Studies \\
    University of Zurich \\
  8057 Zurich, Switzerland \\
  \texttt{\href{mailto:anubhav.gupta@ieu.uzh.ch}{\nolinkurl{anubhav.gupta@ieu.uzh.ch}}} \\
   \And
    Owen L. Petchey
   \\
    Department of Evolutionary Biology and Environmental Studies \\
    University of Zurich \\
  8057 Zurich, Switzerland \\
  \texttt{\href{mailto:owen.petchey@ieu.uzh.ch}{\nolinkurl{owen.petchey@ieu.uzh.ch}}} \\
  }


% tightlist command for lists without linebreak
\providecommand{\tightlist}{%
  \setlength{\itemsep}{0pt}\setlength{\parskip}{0pt}}


% Pandoc citation processing
\newlength{\cslhangindent}
\setlength{\cslhangindent}{1.5em}
\newlength{\csllabelwidth}
\setlength{\csllabelwidth}{3em}
\newlength{\cslentryspacingunit} % times entry-spacing
\setlength{\cslentryspacingunit}{\parskip}
% for Pandoc 2.8 to 2.10.1
\newenvironment{cslreferences}%
  {}%
  {\par}
% For Pandoc 2.11+
\newenvironment{CSLReferences}[2] % #1 hanging-ident, #2 entry spacing
 {% don't indent paragraphs
  \setlength{\parindent}{0pt}
  % turn on hanging indent if param 1 is 1
  \ifodd #1
  \let\oldpar\par
  \def\par{\hangindent=\cslhangindent\oldpar}
  \fi
  % set entry spacing
  \setlength{\parskip}{#2\cslentryspacingunit}
 }%
 {}
\usepackage{calc}
\newcommand{\CSLBlock}[1]{#1\hfill\break}
\newcommand{\CSLLeftMargin}[1]{\parbox[t]{\csllabelwidth}{#1}}
\newcommand{\CSLRightInline}[1]{\parbox[t]{\linewidth - \csllabelwidth}{#1}\break}
\newcommand{\CSLIndent}[1]{\hspace{\cslhangindent}#1}

\usepackage{lineno}
\linenumbers
\usepackage {amsmath}
\setlength\parindent{24pt}
\usepackage{setspace}\doublespacing
\usepackage{booktabs}
\usepackage{longtable}
\usepackage{array}
\usepackage{multirow}
\usepackage{wrapfig}
\usepackage{float}
\usepackage{colortbl}
\usepackage{pdflscape}
\usepackage{tabu}
\usepackage{threeparttable}
\usepackage{threeparttablex}
\usepackage[normalem]{ulem}
\usepackage{makecell}
\usepackage{xcolor}
\begin{document}
\maketitle


\begin{abstract}
\begin{enumerate}
\def\labelenumi{\arabic{enumi})}
\tightlist
\item
  Undersampling can lead to missing trophic interactions in recorded
  food webs, with potential consequences for the perceived functioning
  and stability of the food webs. Undersampling can be compensated for
  by using food web models such as the allometric diet breadth model
  (ADBM) to predict missing links. Simultaneously, models might predict
  links which cannot occur, i.e., real false positives.
\item
  Previous research shows that (i) food web robustness (the inverse of
  the number of secondary extinctions occurring due to primary
  extinctions) increases with connectance (the number of realised
  trophic links divided by the number of possible links), and (ii) that
  model predicted food webs usually have greater connectance than
  observed ones. Thus, we expect that predicted food webs are more
  robust than the observed ones. However, this expectation has never, to
  our knowledge, been tested, nor has the effect size, i.e.~the
  difference in robustness of predicted and observed food web with
  respect to the difference in their connectance, been quantified.
\item
  We fill this research gap by comparing the robustness of observed food
  webs to the robustness of food webs predicted by a model (the ADBM)
  that can account for missing links. We did this for 12 different food
  webs from a wide variety of ecosystems. We used three extinction
  scenarios: random, most connected, and least connected.
\item
  We found, as expected, that the predicted food webs were more robust
  than the observed food webs, which can be attributed to the higher
  connectance of the predicted food webs. On average, for every one unit
  of increase in connectance, we found that the food webs robustness
  increased by 0.52 units for the most connected species extinction
  scenario and by 0.04 units for the random species extinction
  scenarios. On the other hand, we saw no effect or very little effect,
  on average, of increased connectance for the least connected species
  extinction scenario for all except two of the 12 food webs.
\item
  These results show that undersampling can lead to large underestimates
  of food web robustness that can be compensated for by filling in
  missing links with food web models. Also, the differences in the
  structural properties of the model predicted food webs and the
  observed food webs suggest structural properties of a food web could
  be used as summary statistics to fit models such as the ADBM to the
  observed data.
\end{enumerate}
\end{abstract}

\keywords{
    connectance
   \and
    ABC
   \and
    ADBM
   \and
    food web
   \and
    extinction
   \and
    uncertainty
  }

\hypertarget{introduction}{%
\section{Introduction}\label{introduction}}

Anthropogenic changes such as climate change and habitat destruction are
a threat to biodiversity and can lead to food web collapse (Ullah et
al., 2018). This food web collapse is due to the cascades of secondary
extinctions in a food web because of the primary loss of species due to
environmental change (Pimm et al., 2006; C. D. Thomas et al., 2004; J.
A. Thomas et al., 2004). An example of a secondary extinction is when a
consumer goes extinct because its sole resource species goes extinct.
Therefore, research focused on cascading secondary extinctions, also
known as `community viability analysis', has been performed extensively
in the past few decades to quantify how robust food webs are to species
extinction (Berg et al., 2011; Dunne et al., 2002a; Dunne \& Williams,
2009; Ebenman et al., 2004; Ebenman \& Jonsson, 2005). This research
revealed that the rate of collapse of a food web depends on its
structure and complexity (Dunne et al., 2002a; Dunne \& Williams, 2009).

Simulation of primary species loss has been conducted in observed food
webs and model food webs from terrestrial and aquatic ecosystems, where
robustness was measured in terms of secondary extinctions (Dunne et al.,
2002a; Dunne \& Williams, 2009). Primary species loss is considered to
be extinction of a species due to causes external to the food web,
e.g.~overharvesting (Koning \& McIntyre, 2021), introduction of invasive
species (David et al., 2017). A secondary extinction is an extinction
caused by a previous extinction (either primary or secondary). Numerous
studies have used topological criterion for assigning a secondary
extinction to a species, i.e., if extinction is of the last resource
species of a consumer species, then the consumer species suffers
secondary extinction (Dunne et al., 2002a; Dunne \& Williams, 2009;
Sol'e \& Montoya, 2001).

When there are few primary extinctions, and these cause many secondary
extinctions, a food web is said to be fragile or not robust. In
contrast, a food web is said to be robust when there are only a few
secondary extinctions. Several studies have shown that the robustness of
the food webs increases with food web connectance (Dunne et al., 2002b;
Dunne \& Williams, 2009). Additionally, these studies have demonstrated
that the removal (primary extinction) of the most connected species
causes considerably more secondary extinctions than the random removals
of species (Dunne et al., 2002b; Sol'e \& Montoya, 2001). Simulation
studies like these, which investigate the impact of primary extinctions
in a food web to quantify robustness based on its topological structure,
provide an alternate solution to canonical experiments in natural
ecosystems, which are not possible or very difficult to conduct (Dunne
\& Williams, 2009).

\hypertarget{paragraph-on-how-other-structural-food-web-properties-are-responsible-as-well}{%
\subsection{Paragraph on how other structural food web properties are
responsible as
well}\label{paragraph-on-how-other-structural-food-web-properties-are-responsible-as-well}}

Structural properties other than food web connectance such as the
proportion of basal species and the maximum trophic level of species in
a food web can influence the robustness of a food web (Mendonça et al.,
2022; Riede et al., 2011). For example, a food web with a higher
proportion of basal species on average will be more robust as compared
to otherwise (Mendonça et al., 2022) because extinctions of all the
basal species will lead to collapse of the complete food web. A food web
with a lower maximum trophic level of species on average will be more
robust as compared to otherwise (Riede et al., 2011). This is because a
food web with a lower maximum trophic level will have higher number of
species per trophic level on average as compared to otherwise.

Along with quantifying food web robustness based on its topological
structure, studies such as Williams (2008), Brose et al. (2006), and
Martinez et al. (2006) have quantified robustness based on the abundance
dynamics of a food web. However, the topological approach of quantifying
a food web's robustness only requires the food web structure, whereas
the dynamical approach requires the food web structure and the temporal
dynamics of the abundance of species in that food web. For example,
Williams (2008) combined network structure models with bioenergetic
dynamics models to study the role of food web topology and nonlinear
dynamics on species coexistence in complex ecological networks.

A key assumption of the observed food webs is that they are very well
sampled, i.e.~all the links that in reality can occur are represented.
However, it is known that not all food webs are very well sampled and do
not represent all of the feeding links that can occur (Caron et al.,
2022; Jordano, 2016; Patonai \& Jord'an, 2017). Some rare trophic links
require more sampling effort than others, whereas some trophic links
(sometimes referred to as forbidden links) remain unobserved because of
biological constraints such as spatio-temporal uncoupling, size or
reward mismatching, foraging constraints and physiological-biochemical
constraints, which are irrespective of sufficient sampling effort
(Jordano, 2016). Previous studies such as Caron et al. (2022) and Gupta
et al. (2022) have shown that the predicted food webs from these models
usually have greater connectance than the observed ones. One solution to
compensate for undersampling is using a food web model such as the
Allometric Diet Breadth Model (ADBM) (Petchey et al., 2008) (reference
(gut content paper)) to predict the missing links and then measure the
robustness of the predicted food web. The ADBM is a mechanistic model
constructed using foraging rules based on the body sizes of prey and
predator, where trophic interactions satisfying those rules would be
predicted by the model, which are perhaps not observed because those
interactions are rare. However, this solution is not infallible, as it
is likely that the food web model might still miss some links and also
may predict some links that could not, in fact occur.

In our study, we investigate the topological robustness of the ADBM
predicted food webs and compare it to that of the observed food webs. We
achieved this by simulating primary species loss in 12 food webs
predicted by the model to quantify secondary loss of extinctions and
compare the robustness of food webs predicted by the model to the
robustness of the observed food webs. We expect the model predicted food
webs to be more robust than the observed food webs, and for the greater
robustness to be related to the amount by which the model predicts
greater connectance compared to that of the observed food webs. We also
examined how different structural food web properties other than the
connectance can influence this difference in the perceived robustness.
In order to do that, we define our null hypothesis to be: (H1) The
missing links that are filled in by the model increase robustness
because they increase connectance; and our alternate hypothesis to be:
(H2) Although the model fills in missing links, it also changes other
structural properties and these outweigh the effects of connectance on
robustness.

\hypertarget{materials-and-methods}{%
\section{Materials and methods}\label{materials-and-methods}}

In the upcoming sections, we present a detailed account of the
implementation of the simulation of primary extinctions for three
different extinction scenarios on 12 food webs predicted by the ADBM
from a wide variety of ecosystems and compute the resultant secondary
extinctions. We then describe a robustness metric of those predicted
food webs and how we compare them to the properties of the food webs.

\hypertarget{allometric-diet-breadth-model-adbm}{%
\subsection{Allometric Diet Breadth Model
(ADBM)}\label{allometric-diet-breadth-model-adbm}}

The allometric diet breadth model (ADBM) is based on optimal foraging
theory, specifically the contingency model (MacArthur \& Pianka, 1966).
We chose this model because it is known to predict greater connectance
in the predicted food webs compared to the observed food webs (Gupta et
al., 2022). The ADBM predicts the set of prey species a consumer should
feed upon to maximise its rate of energy intake (Petchey et al., 2008).
The foraging variables used in the model are the energy content of prey,
handling times of the predator on prey, space clearance rate, i.e.~how
fast a predator searches space, and prey densities. Each of these
variables is derived from the allometric scaling relationship using the
body sizes of species. More details on the foraging rules defined in the
ADBM and ADBM's predictive power across different food webs can be found
in open access Petchey et al. (2008).

\hypertarget{food-web-data}{%
\subsection{Food web data}\label{food-web-data}}

The observed food webs that we fit the ADBM to belong to marine,
freshwater and terrestrial ecosystems (Table \ref{fig:tab_1}). We
considered these food webs because they belong to diverse ecosystems and
follow FAIR (Findable Accessible Interoperable Reusable) principles
(Wilkinson et al., 2016). These food webs contain primary producers,
herbivores, carnivores, parasites, and parasitoids and various feeding
interactions, including predation, herbivory, bacterivory, parasitism
and pathogenic. The observed connectance of these food webs varies from
0.03 to 0.24 (Links/Species\(^2\)), and the number of species varies
from 29 to 239. The goodness of fit of the ADBM's predictions depends on
the interaction types in the food webs. For example, those with
size-structured interactions, such as herbivory in aquatic ecosystems,
are better predicted compared to less size-structured ones, such as
parasitoids and terrestrial herbivory ones (Petchey et al., 2008).

All food webs with one exception (Broadstone Stream) was available only
at the species level, i.e.~with information about interactions between
species and the body size of species. We use the term ``species'' in
this study to indicate a ``node'' in a food web in which nodes are
connected by trophic interactions, and nodes are a collection of
individuals that share links. These species/nodes are not always
taxonomic species, but can be broader taxonomic ranks.

In contrast, the Broadstone Stream food web data contained interactions
between individuals and the individual body sizes. Thus, the Broadstone
Stream food web can be constructed by aggregating by either taxonomy or
size (Woodward et al., 2010).

\newgeometry{margin=1cm}
\begin{landscape}\begin{table}

\caption{\label{tab:unnamed-chunk-1}\label{fig:tab_1}Information about the food webs predicted using the ADBM.}
\centering
\resizebox{\linewidth}{!}{
\fontsize{7}{9}\selectfont
\begin{tabular}[t]{>{\raggedright\arraybackslash}p{4cm}|>{\raggedright\arraybackslash}p{3cm}|l|>{\raggedright\arraybackslash}p{2cm}|>{\raggedright\arraybackslash}p{2cm}|>{\raggedright\arraybackslash}p{3cm}}
\hline
Common food web name (Original Publication) & Predation matrix source & General ecosystem & Number of species & Observed connectance (Links/Species$^2$) & 95\% prediction interval of predicted connectance  of the ADBM  (Gupta et al. 2022)\\
\hline
Benguela Pelagic (Yodzis 1998) & Brose et al. (2005) & Marine & 30 & 0.21 & 0.26 - 0.59\\
\hline
Broadstone Stream (taxonomic aggregation) (Woodward and Hildrew 2001; Woodward
et al. 2005) & Brose et al. (2005) & Freshwater & 29 & 0.19 & 0.18 - 0.72\\
\hline
Broom (Memmott et al. 2000) & Brose et al. (2005) & Terrestrial & 60 & 0.03 & 0.12 - 0.89\\
\hline
Capinteria (Lafferty et al. 2006) & Hechinger et al. (2011) & Marine (Salt Marsh) & 88 & 0.08 & 0.11 - 0.80\\
\hline
Caricaie Lakes (Cattin et al. 2004) & Brose et al. (2005) & Freshwater & 158 & 0.05 & 0.11 - 0.81\\
\hline
Grasslands (Dawah et al. 1995) & Brose et al. (2005) & Terrestrial & 65 & 0.03 & 0.03 - 0.44\\
\hline
Mill Stream (Ledger, Edwards, Woodward unpublished) & Brose et al. (2005) & Freshwater & 80 & 0.06 & 0.08 - 0.60\\
\hline
Skipwith Pond (Warren 1989) & Brose et al. (2005) & Freshwater & 71 & 0.07 & 0.17 - 0.90\\
\hline
Small Reef (Opitz 1996 Table 8.6.2) & Alyssa R. Cirtwill and Anna Eklöf (2018) & Marine (Reef) & 239 & 0.06 & 0.07 - 0.66\\
\hline
Tuesday Lake (Jonsson et al. 2005) & Brose et al. (2005) & Freshwater & 73 & 0.08 & 0.09 - 0.57\\
\hline
Ythan (Emmerson and Raffaelli 2004) & Alyssa R. Cirtwill and Anna Eklöf (2018) & Marine (Estuarine) & 85 & 0.04 & 0.13 - 0.84\\
\hline
Broadstone Stream (size aggregation) (Woodward
et al. 2010) & Guy Woodward (2021) & Freshwater & 29 & 0.24 & 0.25 - 0.47\\
\hline
\end{tabular}}
\end{table}
\end{landscape}
\restoregeometry

\hypertarget{model-parameterisation-using-approximate-bayesian-computation}{%
\subsection{Model parameterisation using approximate Bayesian
computation}\label{model-parameterisation-using-approximate-bayesian-computation}}

The ADBM was parameterised using approximate Bayesian computation (ABC),
where a set of parameter values were sampled from a prior distribution.
Then, that set of parameter values was either accepted or rejected based
on how close the predicted food web was to the observed food web using
an accuracy metric -- true skill statistic (TSS). The accepted parameter
values formed a posterior distribution (Fig. 4 and S14-S25 in Gupta et
al. (2022)). Further, prediction intervals of the true skill statistic
and connectance of the predicted food webs were computed (Fig. 5 (a, b)
in Gupta et al. (2022)). In our study, we considered model predicted
food webs where the predicted connectance lay within the 95\% prediction
interval of all model predicted food webs. A detailed explanation of the
parameterisation method can be found in Gupta et al. (2022).

\hypertarget{extinction-scenarios-and-robustness}{%
\subsection{Extinction scenarios and
robustness}\label{extinction-scenarios-and-robustness}}

We implemented the primary species removal method from Dunne \& Williams
(2009) by sequentially removing species using one of the three criteria:
removal of (i) the most connected species, (ii) the least connected
species and (iii) randomly chosen species. The most connected and least
connected criteria are based on species' degree (i.e.~the total number
of links to resources and from consumers). We considered these three
criteria because the random extinction scenario takes into account all
the theoretically possible extinction sequences of species that can
occur in a food web, while the extinction of the most connected species
and least connected species takes into account the two opposite extreme
scenarios. These extinction scenarios have been widely used in studying
species extinctions and the collapse of food webs and other networks
(Albert \& Barab'asi, 2002; J. Dunne et al., 2004; Dunne et al., 2002a;
Dunne \& Williams, 2009; Sol'e \& Montoya, 2001).

Given a primary removal of species in a food web, if any remaining
species lost all of their resource species, or any cannibalistic species
lost all of their resource species except the cannibalistic links, they
are removed from the web, and a secondary extinction was recorded.
Secondary extinctions may cause further secondary extinctions, which
were also checked for and recorded. Once no more secondary extinctions
occurred, then another primary extinction was made of the next
appropriate species depending on the extinction scenario. This process
was carried out until all the species were extinct from the food web.

The robustness (R) of a food web was defined as the proportion of
species subjected to primary removals resulting in extinction (primary
and secondary extinctions) of some specified proportion of the species.
In our study, we use \(R_{50}\), the number of primary extinctions
divided by the total number of species, which results in at least 50\%
of total species loss (J. Dunne et al., 2004; Dunne et al., 2002a; Dunne
\& Williams, 2009; Jonsson et al., 2015). Therefore, if primary
extinctions never cause any secondary extinctions, the food web is
maximally robust and (\(R_{50} = 0.50\)). Whereas in a minimally robust
community (\(R_{50} = 1/S\), where \(S\) is the number of species), the
first primary extinction causes a cascade of secondary extinctions of at
least nearly half of the species in the food web (i.e.~at least
\(S/2 - 1\)).

\hypertarget{simulating-species-extinctions}{%
\subsection{Simulating species
extinctions}\label{simulating-species-extinctions}}

First, we simulated primary species loss in food webs predicted by the
ADBM which had the maximum true skill statistics and compared it to
primary species loss in observed food webs. Second, to take into account
the uncertainty in robustness in the ADBM predicted food webs, we
simulated primary species loss and thereby computed robustness for all
the ADBM predicted food webs corresponding to the 95\% prediction
interval of the predicted connectance. Furthermore, in the case of the
random extinction scenario, we simulated 1000 random extinction
sequences in a single ADBM predicted food web.

\hypertarget{analysis}{%
\subsection{Analysis}\label{analysis}}

In the random extinction scenario, we computed robustness \(R_{50}\) for
all 1000 independent random extinction sequences and calculated the
median as a summary statistics to quantify the average robustness of a
single food web to random extinction. To quantify the effect of
undersampling, i.e.~greater connectance of connectance, we compute the
ratio of the difference in normalised robustness between the ADBM
predicted food webs and observed food webs to the difference in their
normalised connectance, where normalisation was performed by dividing
the variables by their maximum possible values (i.e.~0.5 for \(R_{50}\)
and 1 for connectance). However, we did not perform any statistical
significance test because we work with simulated food webs and
therefore, the p-values of these tests are influenced by the number of
model simulations (White et al., 2014).

\hypertarget{results}{%
\section{Results}\label{results}}

We first compare the robustness of the model food webs against that of
the observed food webs. We then quantify the effect of difference in
their connectance on the difference in their robustness estimates.

The model food webs were more robust than the observed food webs on
average in the most connected and random extinction scenarios (Fig.
\ref{fig:fig_r4} (a, b)). However, there were large variations in the
robustness within the model food webs in the most connected extinction
scenario (Fig. \ref{fig:fig_r4} (a)). For example, the model food webs
for the Caricaie Lakes food web was more robust than the observed food
web on average but had a larger variation in the robustness within the
model food webs compared to other food webs.

The food webs were more robust to the random extinction scenario than
the most connected scenario (Fig. \ref{fig:fig_r4} (a, b)). Small Reef
and Benguela Pelagic food webs had more variations in robustness within
the model food webs as compared to the other food webs (Fig.
\ref{fig:fig_r4} (b)). Skipwith Pond and Broadstone Stream (taxonomic
aggregation) food webs were the most robust (Median \(R_{50} = 0.5\))
for both model and observed food webs. The food webs were more robust to
the random extinction scenario than the most connected scenario (Fig.
\ref{fig:fig_r4} (a, b)). Small Reef and Benguela Pelagic food webs had
more variations in robustness within the model food webs as compared to
the other food webs (Fig. \ref{fig:fig_r4} (b)). Skipwith Pond,
Broadstone Stream (taxonomic aggregation) and Broadstone Stream (size
aggregation) food webs were the most robust (Median \(R_{50} = 0.5\))
for both model and observed food webs. Although there were few less
robust model food webs in the Broadstone Stream (size aggregation) as
shown by the outliers.

In the least connected extinction scenario, the food webs had a very
high robustness (Median \(R_{50} = 0.5\)) for most of the food webs
(Fig. \ref{fig:fig_r4} (c)), however there were some exceptions. The
model food webs for Small Reef and Benguela Pelagic had very low median
robustness. The model food webs for the Benguela Pelagic, Broom and
Capinteria food webs had larger variations in robustness when compared
to that of the others.

\begin{figure}

{\centering \includegraphics[width=450px]{../results/plot_R50_ADBM_vs_obs} 

}

\caption{\label{fig:fig_r4} Robustness comparison between the ADBM predicted food webs and the observed food webs for 12 food webs across different ecosystems. Here, $R_{50}$ is the proportion of species that have to be removed to achieve a total loss of at least 50\% of total species (primary removals and secondary extinctions). Box represent the 25th and 75th percentile; solid diamond represents the median; whisker represents outlier limits; the outlier coefficient used was 1.5. Some points are not visible due to perfect overlap in b and c. Refer to Fig. 7 in the Supplementary Information for a faceted visualisation. The dashed black lines are the 1:1 relationships for reference.}\label{fig:unnamed-chunk-2}
\end{figure}

In all of the food webs except Broadstone Stream (taxonomic aggregation)
and Skipwith Pond, the effect size of connectance on robustness was
positive on average in the most connected extinction scenario (Fig.
\ref{fig:fig_r5} (a)), i.e.~greater connectance had a positive effect on
the robustness. In the random extinction scenario, there was a positive
effect of greater connectance on the robustness for Ythan, Small Reef,
Mill Stream, Grasslands, Caricaie Lakes, Capinteria, Broom and Benguela
Pelagic (Fig. \ref{fig:fig_r5} (b)). However, the effect size varied
across the food webs. In the least connected extinction scenario, the
median effect sizes were zero for all the food webs except for Caricaie
Lakes and Capinteria food webs where the effect sizes were very close to
zero and for Benguela Pelagic and Small Reef food webs where the median
effect sizes were negative (Fig. \ref{fig:fig_r5} (c)). However, there
were lots of outlier with effect sizes negative.

\begin{figure}

{\centering \includegraphics[width=450px]{../results/plot_R50_slope} 

}

\caption{\label{fig:fig_r5} Effect size (i.e. ratio of the difference in normalised robustness between ADBM predicted food webs and observed food webs to the difference in their normalised connectance) shown for the 12 food webs. Box represent the 25th and 75th percentile; bold black midline represents the median; whisker represents outlier limits; the outlier coefficient used was 1.5.}\label{fig:unnamed-chunk-3}
\end{figure}

\hypertarget{discussion}{%
\section{Discussion}\label{discussion}}

As expected, the model food webs were more robust than the observed food
webs on average. The considerable variation of the robustness of the
model food webs suggests, however, that undersampling in food webs can
lead to considerable uncertainty in the estimates of food web
robustness, even when a model is used to compensate for undersampling.
Furthermore, as was previously found, the food webs are least robust to
primary extinction of the most connected species compared to that of
least connected and random extinction scenarios on average. Future
development would be to understand how undersampling, i.e.~predicted
greater connectance, influences the stability of the dynamics of the
model food webs against that of the observed food webs and compare it
with the patterns in our study in which extinction occur only by
topological criteria. However, one would expect a decrease in food web
stability with greater connectance (Martinez et al., 2006; May, 1972).

As mentioned, the robustness of the ADBM predicted food webs was higher
than that of the observed food webs on average (Fig. \ref{fig:fig_r4})
for all of the 12 food web ecosystems (with some exceptions). This is
likely due to the greater connectance of the ADBM predicted food webs as
compared to that of the observed food webs because a species in a food
web with a higher connectance has, on average, more trophic links as
compared to a food web with a lower connectance (Fig. \ref{fig:fig_r5}).
Our study suggests that it is important to consider undersampling in
observed food webs when computing their robustness.

\hypertarget{a-paragraph-on-the-implication-of-our-study}{%
\subsection{A paragraph on the implication of our
study}\label{a-paragraph-on-the-implication-of-our-study}}

\begin{itemize}
\tightlist
\item
  On the uncertainty in the ADBM estimates
\item
  On the limitation of a food web model
\item
  On undersampling
\end{itemize}

Our study also depicts that the uncertainty in the model predictions can
have a strong influence on the perceived robustness of the observed food
webs. The large variations in the robustness of the observed food webs
suggest why it is crucial to incorporate uncertainty in the food web
predictions. At the same time, our study also quantifies the impact of
undersampling on the perceived robustness of the observed food webs and
thus influencing the perceived functioning and stability of the food
webs (reference). This strongly suggest why it is crucial to incorporate
the influence of undersampling.

Contrary to general expectations (Dunne et al., 2002b), food web
robustness did not always increase with the connectance (Fig.
\ref{fig:fig_r5}). For example, the Benguela Pelagic and Small Reef ADBM
predicted food webs were surprisingly less robust to primary extinctions
on average in the least connected extinction scenario compared to the
observed food webs (Fig. \ref{fig:fig_r4} (c) and \ref{fig:fig_r5} (c)).
In these two food webs, the extinction of the least connected species
could cause an almost complete, or complete collapse of the food web. We
suspect this is because the ADBM predicted food webs have a lower
proportion of basal species when compared to that of the observed food
webs (Fig: 6 (a) in Gupta et al. (2022)). As a result, these low-degree
basal species are the ones to be removed at an early stage in the
deletion sequence, thereby resulting in an earlier food web collapse in
the ADBM predicted food web as compared to that of the observed food web
(Fig: \ref{fig:fig_r3} (a) and (j)). This suggests that the greater
connectance predicted by the ADBM resulted in a more robust food web on
average. However, differences in the predicted food web properties, such
as a lower proportion of basal species and higher maximum trophic level
(Fig: \ref{fig:fig_a1}) when compared to that of the predicted food webs
counteracted that effect and led to reduced robustness. On average, a
consumer in a food web with a higher maximum trophic level would have
fewer resources and be more susceptible to extinction than a consumer in
a food web with a lower maximum trophic level (Binzer et al., 2011).
This suggests that food web properties other than connectance play an
important role in determining a food web's robustness and, therefore,
should also be taken into account (Binzer et al., 2011; Mendonça et al.,
2022; Riede et al., 2011).

As with any food web model, we expect that there are real false
positives in the food webs predicted by the ADBM. Real false positive
means that the food web model predicts a link between two species that
can never interact (The other type of false positive is when the model
predicts a link that was not observed but could have been observed if
the food web was sampled enough. In this case, further sampling should
result in the link being observed and a change from false positive to
true positive.). Firstly, this may be because the ADBM uses only body
size as a trait. A trait uncorrelated with the body size may be
influential in determining the interaction between two species (Gupta et
al., 2022). Secondly, the ADBM can only predict diets that are
contiguous with respect to the size of the prey. I.e. it cannot predict
that the consumer will consume prey of size 1 and 3, and not consume
prey of size 2. However, it is important to note that observed diets are
not always contiguous when prey are ordered by their size due to some
ecological differences in how predator species choose their prey (Caron
et al., 2022). Hence, it would be intriguing to extend our study to use
other food web models based on size-based rules, such as Gravel et al.
(2013) and Vagnon et al. (2021), to understand if the results are
dependent on the decision of model selection. We expect to get a similar
result in a size-based deterministic model but a different result,
i.e.~lower robustness in a size-based stochastic model as compared to
the ADBM because the latter can take into account non-contiguity in
predator diets (Williams et al., 2010). It would also be interesting to
use food web models not based on body size, such as Cattin et al. (2004)
and Allesina et al. (2008). We expect to have a difference in results
based on whether the trophic interactions in the food webs are governed
by size-structured rules or not.

It would be intriguing to know if this difference in connectance has a
similar influence on the dynamical stability of the food webs as well.
Hence, a prospect could be to use a dynamical model (for example, the
bioenergetic food web model by Brose et al. (2006)) to model the
temporal dynamics of the ADBM predicted food webs. We expect that the
greater connectance will lead to reduced dynamical stability in the ADBM
predicted food web compared to that of the predicted food web. The
difference in stability will be linearly related to the difference in
connectance because Martinez et al. (2006) has shown that food web
stability linearly decreases with connectance.

Since the ADBM predicted food webs have a lower proportion of basal
species and a higher maximum trophic level as compared to that of the
observed food webs (Fig: 6 (a) in Gupta et al. (2022) and Fig.
\ref{fig:fig_a1} in Supplementary Information), it would be interesting
to use these properties as summary statistics to parameterise the ADBM
and investigate how that influences the difference in the robustness
between the ADBM predicted and the observed food webs. We would expect a
more highly constrained predicted food web structure, lower variation in
robustness, and a greater apparent influence of connectance on
robustness.

We have used a food web model to compensate for undersampling in
recorded food webs and thereby quantified the influence of missing
links, i.e.~greater connectance on the topological robustness of 12 food
webs from various ecosystems. We found that the greater connectance can
have a large impact on the robustness of the food webs while at the same
time producing large variations in robustness among the predicted food
webs. Furthermore, differences in other structural food web properties
between the ADBM predicted food webs and the observed food webs are also
responsible.

\hypertarget{acknowledgements}{%
\section{Acknowledgements}\label{acknowledgements}}

This work was supported by the University Research Priority Program
Global Change and Biodiversity (Grant number: U-704-04-11) of the
University of Zurich. We thank the Petchey group members for their
valuable suggestions in the manuscript.

\hypertarget{conflict-of-interest}{%
\section{Conflict of interest}\label{conflict-of-interest}}

None declared

\hypertarget{author-contributions}{%
\section{Author contributions}\label{author-contributions}}

\textbf{Anubhav Gupta:} Conceptualisation; Data curation; Formal
analysis; Investigation; Methodology; Project administration; Software;
Validation; Writing -- original draft; Writing -- review and editing.
\textbf{Owen L. Petchey:} Conceptualization; Funding acquisition;
Resources; Supervision; Writing -- review \& editing.

\hypertarget{data-accessibility-statement}{%
\section{Data Accessibility
Statement}\label{data-accessibility-statement}}

All the data used in this study was collected in other studies and is
openly available. We list those studies and the open access source in
Table \ref{fig:tab_1}. The complete code used in the analysis is
available in the repository
\url{https://doi.org/10.5281/zenodo.7180835}.

\hypertarget{supplementary-information}{%
\section{Supplementary Information}\label{supplementary-information}}

\begin{figure}[H]

{\centering \includegraphics[width=450px]{../results/plot_max_tl_ADBM_vs_emp} 

}

\caption{\label{fig:fig_a1} The maximum trophic level of ADBM predicted food webs plotted against that of the observed food webs. Box represent the 25th and 75th percentile; bold midline represents the median; whisker represents outlier limits; the outlier coefficient used was 1.5. The dashed black lines are the 1:1 relationships for reference.}\label{fig:unnamed-chunk-4}
\end{figure}

\begin{figure}[H]

{\centering \includegraphics[width=450px]{../results/plot_R50_ADBM_vs_obs_ra_lc_facet} 

}

\caption{\label{fig:fig_a2} Robustness comparison between the ADBM predicted food webs and the observed food webs for 12 food webs across different ecosystems for random and least connected extinction scenarios. Here, $R_{50}$ is the proportion of species that have to be removed to achieve a total loss of at least 50\% of total species (primary removals and secondary extinctions). Box represent the 25th and 75th percentile; solid diamond represents the median; whisker represents outlier limits; the outlier coefficient used was 1.5. The dashed black lines are the 1:1 relationships for reference.}\label{fig:unnamed-chunk-5}
\end{figure}

\hypertarget{references}{%
\section*{References}\label{references}}
\addcontentsline{toc}{section}{References}

\hypertarget{refs}{}
\begin{CSLReferences}{1}{0}
\leavevmode\vadjust pre{\hypertarget{ref-albertStatisticalMechanicsComplex2002}{}}%
Albert, R., \& Barab'asi, A.-L. (2002). Statistical mechanics of complex
networks. \emph{Reviews of Modern Physics}, \emph{74}(1), 47--97.
\url{https://doi.org/10.1103/RevModPhys.74.47}

\leavevmode\vadjust pre{\hypertarget{ref-allesinaGeneralModelFood2008}{}}%
Allesina, S., Alonso, D., \& Pascual, M. (2008). A {General Model} for
{Food Web Structure}. \emph{Science}, \emph{320}(5876), 658--661.
\url{https://doi.org/10.1126/science.1156269}

\leavevmode\vadjust pre{\hypertarget{ref-bergUsingSensitivityAnalysis2011}{}}%
Berg, S., Christianou, M., Jonsson, T., \& Ebenman, B. (2011). Using
sensitivity analysis to identify keystone species and keystone links in
size-based food webs. \emph{Oikos}, \emph{120}(4), 510--519.
\url{https://doi.org/10.1111/j.1600-0706.2010.18864.x}

\leavevmode\vadjust pre{\hypertarget{ref-binzerSusceptibilitySpeciesExtinctions2011}{}}%
Binzer, A., Brose, U., Curtsdotter, A., Eklöf, A., Rall, B. C., Riede,
J. O., \& de Castro, F. (2011). The susceptibility of species to
extinctions in model communities. \emph{Basic and Applied Ecology},
\emph{12}(7), 590--599. \url{https://doi.org/10.1016/j.baae.2011.09.002}

\leavevmode\vadjust pre{\hypertarget{ref-broseAllometricScalingEnhances2006}{}}%
Brose, U., Williams, R. J., \& Martinez, N. D. (2006). Allometric
scaling enhances stability in complex food webs. \emph{Ecology Letters},
\emph{9}(11), 1228--1236.
\url{https://doi.org/10.1111/j.1461-0248.2006.00978.x}

\leavevmode\vadjust pre{\hypertarget{ref-caronAddressingEltonianShortfall}{}}%
Caron, D., Maiorano, L., Thuiller, W., \& Pollock, L. J. (2022).
Addressing the {Eltonian} shortfall with trait-based interaction models.
\emph{Ecology Letters}, \emph{n/a}(n/a).
\url{https://doi.org/10.1111/ele.13966}

\leavevmode\vadjust pre{\hypertarget{ref-cattinPhylogeneticConstraintsAdaptation2004}{}}%
Cattin, M.-F., Bersier, L.-F., Banašek-Richter, C., Baltensperger, R.,
\& Gabriel, J.-P. (2004). Phylogenetic constraints and adaptation
explain food-web structure. \emph{Nature}, \emph{427}(6977, 6977),
835--839. \url{https://doi.org/10.1038/nature02327}

\leavevmode\vadjust pre{\hypertarget{ref-davidImpactsInvasiveSpecies2017}{}}%
David, P., Th'ebault, E., Anneville, O., Duyck, P.-F., Chapuis, E., \&
Loeuille, N. (2017). Impacts of {Invasive Species} on {Food Webs}. In
\emph{Advances in {Ecological Research}} (Vol. 56, pp. 1--60).
{Elsevier}. \url{https://doi.org/10.1016/bs.aecr.2016.10.001}

\leavevmode\vadjust pre{\hypertarget{ref-dunneCascadingExtinctionsCommunity2009}{}}%
Dunne, J. A., \& Williams, R. J. (2009). Cascading extinctions and
community collapse in model food webs. \emph{Philosophical Transactions
of the Royal Society B: Biological Sciences}, \emph{364}(1524),
1711--1723. \url{https://doi.org/10.1098/rstb.2008.0219}

\leavevmode\vadjust pre{\hypertarget{ref-dunne2002network}{}}%
Dunne, J. A., Williams, R. J., \& Martinez, N. D. (2002a). Network
structure and biodiversity loss in food webs: Robustness increases with
connectance. \emph{Ecology Letters}, \emph{5}(4), 558--567.

\leavevmode\vadjust pre{\hypertarget{ref-dunneNetworkStructureBiodiversity2002}{}}%
Dunne, J. A., Williams, R. J., \& Martinez, N. D. (2002b). Network
structure and biodiversity loss in food webs: Robustness increases with
connectance. \emph{Ecology Letters}, \emph{5}(4), 558--567.
\url{https://doi.org/10.1046/j.1461-0248.2002.00354.x}

\leavevmode\vadjust pre{\hypertarget{ref-dunne2004}{}}%
Dunne, J., Williams, R., \& Martinez, N. (2004). Network structure and
robustness of marine food webs. \emph{Marine Ecology Progress Series},
\emph{273}, 291--302. \url{https://doi.org/10.3354/meps273291}

\leavevmode\vadjust pre{\hypertarget{ref-ebenmanUsingCommunityViability2005}{}}%
Ebenman, B., \& Jonsson, T. (2005). Using community viability analysis
to identify fragile systems and keystone species. \emph{Trends in
Ecology \& Evolution}, \emph{20}(10), 568--575.
\url{https://doi.org/10.1016/j.tree.2005.06.011}

\leavevmode\vadjust pre{\hypertarget{ref-ebenmanCOMMUNITYVIABILITYANALYSIS2004}{}}%
Ebenman, B., Law, R., \& Borrvall, C. (2004). {COMMUNITY VIABILITY
ANALYSIS}: {THE RESPONSE OF ECOLOGICAL COMMUNITIES TO SPECIES LOSS}.
\emph{Ecology}, \emph{85}(9), 2591--2600.
\url{https://doi.org/10.1890/03-8018}

\leavevmode\vadjust pre{\hypertarget{ref-gravelInferringFoodWeb2013a}{}}%
Gravel, D., Poisot, T., Albouy, C., Velez, L., \& Mouillot, D. (2013).
Inferring food web structure from predator--prey body size
relationships. \emph{Methods in Ecology and Evolution}, \emph{4}(11),
1083--1090. \url{https://doi.org/10.1111/2041-210X.12103}

\leavevmode\vadjust pre{\hypertarget{ref-guptaSimultaneouslyEstimatingFood2022}{}}%
Gupta, A., Furrer, R., \& Petchey, O. L. (2022). Simultaneously
estimating food web connectance and structure with uncertainty.
\emph{Ecology and Evolution}, \emph{12}(3), e8643.
\url{https://doi.org/10.1002/ece3.8643}

\leavevmode\vadjust pre{\hypertarget{ref-jonssonReliabilityR50Measure2015}{}}%
Jonsson, T., Berg, S., Pimenov, A., Palmer, C., \& Emmerson, M. (2015).
The reliability of {R50} as a measure of vulnerability of food webs to
sequential species deletions. \emph{Oikos}, \emph{124}(4), 446--457.
\url{https://doi.org/10.1111/oik.01588}

\leavevmode\vadjust pre{\hypertarget{ref-jordanoSamplingNetworksEcological2016}{}}%
Jordano, P. (2016). Sampling networks of ecological interactions.
\emph{Functional Ecology}, \emph{30}(12), 1883--1893.
\url{https://doi.org/10.1111/1365-2435.12763}

\leavevmode\vadjust pre{\hypertarget{ref-koningGrassrootsReservesRescue2021}{}}%
Koning, A. A., \& McIntyre, P. B. (2021). Grassroots reserves rescue a
river food web from cascading impacts of overharvest. \emph{Frontiers in
Ecology and the Environment}, \emph{19}(3), 152--158.
\url{https://doi.org/10.1002/fee.2293}

\leavevmode\vadjust pre{\hypertarget{ref-macarthur1966}{}}%
MacArthur, R. H., \& Pianka, E. R. (1966). On optimal use of a patchy
environment. \emph{The American Naturalist}, \emph{100}(916), 603--609.
\url{https://www.jstor.org/stable/2459298}

\leavevmode\vadjust pre{\hypertarget{ref-martinezDiversityComplexityPersistence}{}}%
Martinez, N. D., Williams, R. J., \& Dunne, J. A. (2006).
\emph{Diversity, {Complexity}, and {Persistence} in {Large Model
Ecosystems}}. 24.

\leavevmode\vadjust pre{\hypertarget{ref-mayWillLargeComplex1972}{}}%
May, R. M. (1972). Will a {Large Complex System} be {Stable}?
\emph{Nature}, \emph{238}(5364), 413.
\url{https://doi.org/10.1038/238413a0}

\leavevmode\vadjust pre{\hypertarget{ref-mendoncaRobustnessTemperateTropical2022}{}}%
Mendonça, V., Madeira, C., Dias, M., Flores, A., \& Vinagre, C. (2022).
Robustness of temperate versus tropical food webs: Comparing species
trait-based sequential deletions. \emph{Marine Ecology Progress Series},
\emph{691}, 19--28. \url{https://doi.org/10.3354/meps14062}

\leavevmode\vadjust pre{\hypertarget{ref-patonaiAggregationIncompleteFood2017}{}}%
Patonai, K., \& Jord'an, F. (2017). Aggregation of incomplete food web
data may help to suggest sampling strategies. \emph{Ecological
Modelling}, \emph{352}, 77--89.
\url{https://doi.org/10.1016/j.ecolmodel.2017.02.024}

\leavevmode\vadjust pre{\hypertarget{ref-petchey2008}{}}%
Petchey, O. L., Beckerman, A. P., Riede, J. O., \& Warren, P. H. (2008).
Size, foraging, and food web structure. \emph{Proceedings of the
National Academy of Sciences}, \emph{105}(11), 4191--4196.
\url{https://doi.org/10.1073/pnas.0710672105}

\leavevmode\vadjust pre{\hypertarget{ref-pimmHumanImpactsRates2006}{}}%
Pimm, S., Raven, P., Peterson, A., H. Şekercioğlu, Çağan, \& Ehrlich, P.
R. (2006). Human impacts on the rates of recent, present, and future
bird extinctions. \emph{Proceedings of the National Academy of
Sciences}, \emph{103}(29), 10941--10946.
\url{https://doi.org/10.1073/pnas.0604181103}

\leavevmode\vadjust pre{\hypertarget{ref-riedeSizebasedFoodWeb2011}{}}%
Riede, J. O., Binzer, A., Brose, U., de Castro, F., Curtsdotter, A.,
Rall, B. C., \& Eklöf, A. (2011). Size-based food web characteristics
govern the response to species extinctions. \emph{Basic and Applied
Ecology}, \emph{12}(7), 581--589.
\url{https://doi.org/10.1016/j.baae.2011.09.006}

\leavevmode\vadjust pre{\hypertarget{ref-soleComplexityFragilityEcological2001}{}}%
Sol'e, R. V., \& Montoya, M. (2001). Complexity and fragility in
ecological networks. \emph{Proceedings of the Royal Society of London.
Series B: Biological Sciences}, \emph{268}(1480), 2039--2045.
\url{https://doi.org/10.1098/rspb.2001.1767}

\leavevmode\vadjust pre{\hypertarget{ref-thomasExtinctionRiskClimate2004}{}}%
Thomas, C. D., Cameron, A., Green, R. E., Bakkenes, M., Beaumont, L. J.,
Collingham, Y. C., Erasmus, B. F. N., de Siqueira, M. F., Grainger, A.,
Hannah, L., Hughes, L., Huntley, B., van Jaarsveld, A. S., Midgley, G.
F., Miles, L., Ortega-Huerta, M. A., Townsend Peterson, A., Phillips, O.
L., \& Williams, S. E. (2004). Extinction risk from climate change.
\emph{Nature}, \emph{427}(6970, 6970), 145--148.
\url{https://doi.org/10.1038/nature02121}

\leavevmode\vadjust pre{\hypertarget{ref-thomasComparativeLossesBritish2004}{}}%
Thomas, J. A., Telfer, M. G., Roy, D. B., Preston, C. D., Greenwood, J.
J. D., Asher, J., Fox, R., Clarke, R. T., \& Lawton, J. H. (2004).
Comparative {Losses} of {British Butterflies}, {Birds}, and {Plants} and
the {Global Extinction Crisis}. \emph{Science}, \emph{303}(5665),
1879--1881. \url{https://doi.org/10.1126/science.1095046}

\leavevmode\vadjust pre{\hypertarget{ref-ullah2018}{}}%
Ullah, H., Nagelkerken, I., Goldenberg, S. U., \& Fordham, D. A. (2018).
Climate change could drive marine food web collapse through altered
trophic flows and cyanobacterial proliferation. \emph{PLOS Biology},
\emph{16}(1), e2003446.
\url{https://doi.org/10.1371/journal.pbio.2003446}

\leavevmode\vadjust pre{\hypertarget{ref-vagnonAllometricNicheModel2021}{}}%
Vagnon, C., Cattan'eo, F., Goulon, C., Grimardias, D., Guillard, J., \&
Frossard, V. (2021). An allometric niche model for species interactions
in temperate freshwater ecosystems. \emph{Ecosphere}, \emph{12}(3),
e03420. \url{https://doi.org/10.1002/ecs2.3420}

\leavevmode\vadjust pre{\hypertarget{ref-whiteEcologistsShouldNot2014}{}}%
White, J. W., Rassweiler, A., Samhouri, J. F., Stier, A. C., \& White,
C. (2014). Ecologists should not use statistical significance tests to
interpret simulation model results. \emph{Oikos}, \emph{123}(4),
385--388. \url{https://doi.org/10.1111/j.1600-0706.2013.01073.x}

\leavevmode\vadjust pre{\hypertarget{ref-wilkinsonFAIRGuidingPrinciples2016}{}}%
Wilkinson, M. D., Dumontier, M., Aalbersberg, Ij. J., Appleton, G.,
Axton, M., Baak, A., Blomberg, N., Boiten, J.-W., da Silva Santos, L.
B., Bourne, P. E., Bouwman, J., Brookes, A. J., Clark, T., Crosas, M.,
Dillo, I., Dumon, O., Edmunds, S., Evelo, C. T., Finkers, R., \ldots{}
Mon, B. (2016). The {FAIR Guiding Principles} for scientific data
management and stewardship. \emph{Scientific Data}, \emph{3}(1, 1),
160018. \url{https://doi.org/10.1038/sdata.2016.18}

\leavevmode\vadjust pre{\hypertarget{ref-williamsEffectsNetworkDynamical2008}{}}%
Williams, R. J. (2008). Effects of network and dynamical model structure
on species persistence in large model food webs. \emph{Theoretical
Ecology}, \emph{1}(3), 141--151.
\url{https://doi.org/10.1007/s12080-008-0013-5}

\leavevmode\vadjust pre{\hypertarget{ref-williamsProbabilisticNicheModel2010}{}}%
Williams, R. J., Anandanadesan, A., \& Purves, D. (2010). The
{Probabilistic Niche Model Reveals} the {Niche Structure} and {Role} of
{Body Size} in a {Complex Food Web}. \emph{PLoS ONE}, \emph{5}(8),
e12092. \url{https://doi.org/10.1371/journal.pone.0012092}

\leavevmode\vadjust pre{\hypertarget{ref-woodwardChapterIndividualBasedFood2010}{}}%
Woodward, G., Blanchard, J., Lauridsen, R. B., Edwards, F. K., Jones, J.
I., Figueroa, D., Warren, P. H., \& Petchey, O. L. (2010). Chapter 6 -
{Individual-Based Food Webs}: {Species Identity}, {Body Size} and
{Sampling Effects}. In G. Woodward (Ed.), \emph{Advances in {Ecological
Research}} (Vol. 43, pp. 211--266). {Academic Press}.
\url{https://doi.org/10.1016/B978-0-12-385005-8.00006-X}

\end{CSLReferences}

\bibliographystyle{biblatex}
\bibliography{bibliography.bib}


\end{document}
